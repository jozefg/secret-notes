\documentclass[reqno]{amsart}
\usepackage{preamble,categories,macros}
\usepackage{locallabel,pftools}
\usepackage{jmsdelim}
\usepackage{jon-tikz}

\addbibresource{refs.bib}

\NewDocumentCommand{\Ind}{og}{\IfValueT{#1}{#1\textnormal{-}}\mathbf{Ind}\IfValueT{#2}{(#2)}}
\NewDocumentCommand{\Cont}{ommg}{[#2,#3]_{\IfValueT{#1}{#1\textnormal{-}}\mathsf{cont}\IfValueT{#4}{\mathsf{,#4}}}}
\NewDocumentCommand{\CoCont}{ommg}{[#2,#3]_{\IfValueT{#1}{#1\textnormal{-}}\mathsf{cocont}\IfValueT{#4}{\mathsf{,#4}}}}
\NewDocumentCommand{\PRES}{}{\mathbf{Pres}}
\NewDocumentCommand{\LOGOS}{}{\mathbf{Logos}}
\NewDocumentCommand{\TOPOS}{}{\mathbf{Topos}}

\title{The Tensor Product of Locally Presentable Categories and Grothendieck Topoi}
\author{Daniel Gratzer}
\email{gratzer@cs.au.dk}
\address{Aarhus, Denmark}
\date{\today}

\begin{document}
\begin{abstract}
  This note contains a definition and few elementary results for a symmetric tensor product of
  topoi. No single source seems to collect these basic definitions into one place, so we focus on
  establishing the core properties of this tensor rather than accumulating a great many specialized
  results.
\end{abstract}

\maketitle

We rely on \textcite{adamek-rosicky:1994} for the definition of locally presentable categories and
on \textcite{anel:2019} for the definition of a logos. The latter mentions the following
constructions briefly as well as motivation for them, but does not provide a fully worked account of
their basic properties.

\begin{definition}
  \label{def:tensor}
  Given two locally presentable categories $\EE$ and $\FF$, set
  $\EE \otimes \FF = \Cont[\kappa]{\Op{\EE}}{\FF}$
\end{definition}

This definition is utterly inscrutable on first encounter, but let us observe one crucial fact
which explains its existence:

\begin{theorem}
  \label{def:stable}
  $\otimes$ is stable under equivalence in both arguments.
\end{theorem}

As we shall see, there are several different ways we could have defined $\otimes$, but these more
intuitive definitions rely on particular presentations of $\EE$ and $\FF$ as
$\Ind[\kappa](A)$ or (when we are working with Grothendieck topoi) as $\SH{\CC}[J]$. These
presentations are not canonical, so a definition directly reliant on them would be difficult to
prove stable under equivalence.

Perhaps the most important case to study in order to grasp the behavior of \cref{def:tensor} is when
$\EE$ and $\FF$ are both of the form $\PSH{\CC_i}$:
\begin{example}
  $\PSH{\CC_0} \otimes \PSH{\CC_1} \Equiv \PSH{\CC_0 \times \CC_1}$.
\end{example}
\begin{proof}
  First, observe that $\Cont{\Op{\PSH{\CC_0}}}{\PSH{\CC_1}}$ is equivalent to
  $\Manual{[\PSH{\CC_0},\Op{\PSH{\CC_1}}]^{\mathsf{op}}_{\mathsf{cocont}}}$. We can reduce this last term to
  $\Op{\Hom{\CC_0}{\Op{\PSH{\CC_1}}}} \Equiv \Hom{\Op{\CC_0}}{\PSH{\CC_1}}$ using the universal
  property of $\PSH{-}$ as a free cocompletion. Now currying, we have the conclusion:
  \[
    \Hom{\Op{\CC_0}}{\PSH{\CC_1}}
    \Equiv \Hom{\Op{(\CC_0 \times \CC_1)}}{\SET}
    \Equiv \PSH{\CC_0 \times \CC_1} \qedhere
  \]
\end{proof}

As a first proper result, we show that if $\EE$ and $\FF$ are locally presentable (resp. logoi) then
so too is $\EE \otimes \FF$.

\begin{theorem}
  \label{thm:locally-presentable}
  If $\EE$ and $\FF$ are locally presentable, so too is $\EE \otimes \FF$.
\end{theorem}
\begin{proof}
  First, observe that $\EE = \Ind[\kappa]{\CC}$, for a small category $\CC$ which is cocomplete for
  $\kappa$ limits. Therefore, $\Cont{\Op{\EE}}{\FF} \Equiv \Cont[\kappa]{\Op{\CC}}{\FF}$, using the
  universal property of $\Ind[\kappa]$ completions. Without loss of generality, we may assume that
  $\FF = \Ind[\kappa]{\DD}$ (raising the index of $\EE$ if necessary). Since limits in
  $\Ind[\kappa]{\DD}$ are formed pointwise, we have
  $\Cont[\kappa]{\Op{\CC}}{\FF} \Equiv \Cont[\kappa]{\Op{\CC} \times \Op{\DD}}{\SET}$. Now, this is
  in turn equivalent to $\Cont[\kappa]{\Op{(\CC \times \DD)}}{\SET}$ so it suffices to show that
  this is locally presentable, but this follows from the representation theorem of locally
  presentable categories~\parencite[Theorem 1.46]{adamek-rosicky:1994}.
\end{proof}

\begin{theorem}
  \label{thm:logos}
  If $\EE$ and $\FF$ are logoi, so too is $\EE \otimes \FF$.
\end{theorem}
\begin{proof}
  Again, pick presentations of $\EE$ and $\FF$ as $\SH{\CC}[J]$ and $\SH{\DD}[K]$. First, we observe
  that $\Cont{\Op{\EE}}{\FF} \Equiv \Cont[J]{\Op{\CC}}{\FF}$ with the right-hand side being the set
  of functor $\Mor{\Op{\CC}}{\FF}$ which send a sieve of $J$ to a limit diagram in $\FF$, a detailed
  explanation of this refinement of the standard universal property of free cocompletions is given
  in \textcite[Corollary VII.7.4]{maclane-moerdijk:1992}.\footnote{The notation here is horribly
    clashing with many standard sources, including \parencite{maclane-moerdijk:1992}. However, as
    noted by Johnstone, the notation is uniformly confusing in any case.} Now, we unfold $\FF$ in
  order to apply currying, just as in \cref{thm:locally-presentable} and consider
  $\Cont[J]{\Op{\CC}}{\SH{\DD}[K]}$. Limits are taken pointwise, so this is some subset of functors
  $\Mor{\Op{(\CC \times \DD)}}{\SET}$. In fact, with the observation that sheaves are precisely
  functors which take sieves to limit diagrams of sets and that limits commute with limits, this is
  precisely the set of sieves for $J \times K$, where
  \[
    (J \times K)(C,D) = \bigcup_{\substack{S \in J(C) \\ T \in K(D)}} \{\Mor[(f,g)]{(C',D')}{(C,D)} \mid f \in S(C') \land g \in T(D')\}
  \]
  Therefore $\EE \otimes \FF \Equiv \SH{\CC \times \DD}[J \times K]$ which is clearly a logos.
\end{proof}

We wish to show that these constructions are appropriately functorial, but this is cumbersome to
show directly. Instead, we will establish pseudofunctoriality \emph{en passant} by showing that
$\EE \otimes \FF$ enjoys an appropriate universal property akin to the universal property of the
familiar tensor product from linear algebra.

\begin{theorem}
  \label{thm:locally-presentable-functorial}
  Given the locally presentable categories $\EE$, $\FF$, $\GG$, the subcategory of functor
  $\Hom{\EE \times \FF}{\GG}$ which are cocontinuous in both $\EE$ and $\FF$ is equivalent to the
  category $\CoCont{\EE \otimes \FF}{\GG}$.
\end{theorem}
\begin{proof}
  First, let us observe that $F \in \Hom{\EE \times \FF}{\GG}$ which is cocontinuous in both
  variables is uniquely determined as an element of $\CoCont{\EE}{\CoCont{\FF}{\GG}}$ (and in fact
  this is part of an equivalence of categories). Therefore, it suffices to prove
  $\CoCont{\EE}{\CoCont{\FF}{\GG}} \Equiv \CoCont{\EE \otimes \FF}{\GG}$. We will prove this by
  chaining together a series of equivalences~\parencite{nlab:presentable-infinity-category}:
  \begin{align*}
    \CoCont{\EE}{\CoCont{\FF}{\GG}}
    &\Equiv \CoCont[\kappa]{\CC}{\CoCont{\FF}{\GG}}
    && \text{Presenting $\EE = \Ind[\kappa]{\CC}$}\\
    &\Equiv \Manual{[\Op{\CC}, [\FF,\GG]_{\mathsf{cocont}}^{\mathsf{op}}]_{\kappa\textnormal{-}\mathsf{cont}}^{\mathsf{op}}}\\
    &\Equiv \Manual{[\Op{\CC}, \Cont{\GG}{\FF}{acc}]_{\kappa\textnormal{-}\mathsf{cont}}^{\mathsf{op}}}
    &&\text{Special adjoint functor theorem}\\
    &\Equiv \Manual{[\GG, \Cont[\kappa]{\Op{\CC}}{\FF}]_{\mathsf{cont,acc}}^{\mathsf{op}}}
      \TagEq[$\dagger$]
      \label{eq:commutation}\\
    &\Equiv \Manual{[\GG, \Cont{\Op{\EE}}{\FF}]_{\mathsf{cont,acc}}^{\mathsf{op}}}\\
    &\Equiv \CoCont{\Cont{\Op{\EE}}{\FF}}{\GG}
    && \text{Special adjoint functor theorem}\\
    &\Equiv \CoCont{\EE \otimes \FF}{\GG}
  \end{align*}
  The main subtlety is \cref{eq:commutation}, which requires us to show a certain commutation of
  limits property. In particular, suppose $F \in \Hom{\Op{\CC} \times \GG}{\FF}$. We wish to show
  that the following conditions are equivalent:
  \begin{enumerate}
  \item $F(c,-)$ is continuous and accessible and $F(-,e)$ preserves $\kappa$ limits.
  \item $F(-,e)$ is $\kappa$-continuous, $F(d,-)$ is continuous, and $e \mapsto F(-,e)$ is
    accessible.
  \end{enumerate}
  First, recall that if $\mu > \kappa$, then $\mu$-filtered colimits commute with $\kappa$
  limits. Therefore, showing that $e \mapsto F(-,e)$ is accessible for some $\mu > \kappa$ (as in
  (2.)) is equivalent to showing that $F(c,-)$ is $\mu$-accessible for all $c$. In order to show
  that (1.) is equivalent to (2.), therefore, we must show that if each $F(c,-)$ is
  $\mu_c$-accessible, then there exists a single $\mu$ such that $F(c,-)$ is $\mu$-accessible for
  all $c$. Since $\CC$ is a small category, we may take $\mu = \bigvee_{c} \mu_c$ for the desired
  equivalence.
\end{proof}
\begin{corollary}
  $\otimes$ is the coproduct of locally presentable categories.
\end{corollary}
\begin{corollary}
  There exists a map $\Mor{\EE \times \FF}{\EE \otimes \FF}$ cocontinuous in both variables.
\end{corollary}
\begin{corollary}
  $\otimes$ is pseudofunctorial in both arguments.
\end{corollary}

\begin{theorem}
  \label{thm:logos-functorial}
  Given the logoi $\EE$, $\FF$, $\GG$, the subcategory of functor $\Hom{\EE \times \FF}{\GG}$ which
  are lex and cocontinuous in both $\EE$ and $\FF$ is equivalent to
  $\CoCont{\EE \otimes \FF}{\GG}{lex}$.
\end{theorem}
\begin{proof}
  Using the computations in \cref{thm:logos}, it is sufficient to show to check this assuming that
  $\EE = \SH{\CC}[J]$ and $\FF = \SH{\DD}[K]$. Furthermore, we may assume without loss of generality
  that both $\CC$ and $\DD$ are finitely complete (a corollary of the proof of the Giraud axioms
  capture logoi~\parencite{maclane-moerdijk:1992}). Under these assumptions, a lex cocontinuous functor
  $\Mor[F]{\EE \otimes \FF}{\GG}$ is uniquely determined as a $(J \times K)$-continuous lex functor
  $\Mor[F']{\CC \times \DD}{\GG}$. Furthermore, because $F'$ is left-exact (and in particular
  preserves finite products), it is uniquely determined by an element of
  $\Cont[J]{\CC}{\GG}{lex}$ and $\Cont[K]{\DD}{\GG}{lex}$. From this, the result follows.
\end{proof}
\begin{corollary}
  $\otimes$ is the coproduct of logoi.
\end{corollary}

\printbibliography
\end{document}
