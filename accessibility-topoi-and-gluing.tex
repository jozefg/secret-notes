\documentclass{amsart}
\usepackage{preamble,categories,macros,}
\usepackage{locallabel,pftools}
\usepackage{jmsdelim}
\usepackage{jon-tikz}

\addbibresource{refs.bib}

\title{Accessibility and Artin Gluing}
\author{Daniel Gratzer}
\email{gratzer@cs.au.dk}
\address{Aarhus, Denmark}
\date{\today}

\begin{document}
\begin{abstract}
  This note contains a few proofs of elementary facts in the theory of accessible
  categories. Specifically, we show that comma categories are accessible under the appropriate
  conditions and, as a corollary, derive that gluing along a lex accessible functor between
  Grothendieck toposes gives rise to another Grothendieck topos.
\end{abstract}

\maketitle

\section{Accessible Categories, Accessible Functors}
\label{sec:accessible}

All of these results may be found in \textcite{adamek-rosicky:1994}, we recall some relevant
results here.

\begin{definition}
  \label{def:accessible:accessible-category}
  A category $\EE$ is said to be \emph{$\kappa$-accessible} if the following two conditions hold:
  \begin{enumerate}
  \item $\EE$ has all $\kappa$-filtered limits.
  \item There is a small set of $\kappa$-compact objects of $\EE$ which generates $\EE$ by
    $\kappa$-filtered colimits.
  \end{enumerate}
  We say that $\EE$ is accessible if it is $\kappa$-accessible for some $\kappa$.
\end{definition}

\begin{lemma}
  Any cocomplete accessible category is locally presentable.
\end{lemma}

\begin{definition}
  \label{def:accessible-accessible-functor}
  A functor between $\kappa$-accessible categories is $\kappa$-accessible if it preserves all
  $\kappa$-filtered colimits. Again, we say that $F$ is accessible if $F$ is $\kappa$-accessible for
  some $\kappa$ and its domain and codomain are accessible at this cardinal.
\end{definition}

\begin{definition}
  \label{def:accessible:sharply-larger}
  An regular cardinal $\lambda$ is said to be \emph{sharply larger} than another regular cardinal
  $\kappa$ if for every $\lambda' < \lambda$, the set
  $P_{\kappa} = \{S \in \Pow{\lambda'} \mid \verts{S} < \kappa\}$ has a cofinal subset of
  cardinality strictly less than $\lambda$.
\end{definition}

\begin{lemma}
  \label{lem:accessible:sharp-directed-colim}
  If $\kappa \lhd \lambda$, in each $\kappa$-directed poset every subset with less than $\lambda$
  elements is contained in a $\kappa$-directed subset with less than $\lambda$ elements.
\end{lemma}
\begin{proof}
  Suppose we have such a poset $P$ and a subset $X$ such that $\verts{X} < \lambda$. We wish to
  define $Y$ such that (1.) $Y$ is $\kappa$-directed (2.) $X \subseteq Y$ (3.)
  $\verts{Y} < \lambda$.

  We will define $Y$ as the union of the chain $\bigcup_{i < \kappa} X_i$, where $X_i$ is defined by
  transfinite induction. We set $X_0 = X$ and in all other the limit cases, $X_i$ is the union of
  prior $X_j$s. For the successor cases, suppose we must define $1 + i$. We choose a cofinal subset
  of $Z$ of $X_{i,\kappa} = \{S \in \Pow{X_i} \mid \verts{S} < \kappa\}$, by assumption, this can be
  done such that $Z$ has cardinality strictly smaller than $\lambda$. For each $A \in X_{i,\kappa}$,
  we choose some upper bound $p_A \in P$. We now take the following definition:
  \[
    X_{1 + i} = X_i \cup \bigcup_{A \in Z} (A \cup \{p_A\})
  \]
  By assumption, $X_i$ has cardinality smaller than $\lambda$, as does $Z$ and each
  $A \cup \{p_A\}$. Therefore, using the fact that $\lambda$ is regular, we conclude that
  $X_{1 + i}$ is strictly smaller than $\lambda$. Again using the regularity of $\lambda$, we
  conclude that $Y = \bigcup_i X_i$ has cardinality strictly smaller than $\lambda$. It obviously
  contains $X = X_0$, so it is sufficient to argue that $Y$ is $\kappa$-directed.

  Take any $\kappa$-directed subset $(y_a)_{a \in A} \subseteq Y$. First, observe for any cofinal
  subset $A' \subseteq A$, if $(y_a)_{a \in A'} \subseteq X_i$, then there is an upper-bound for
  this $(y_a)_{a \in A}$ in $Y_{1 + i}$ by definition. Therefore, assume that for each $X_i$, there
  exists some $a_i \subseteq A$ such that $y_{b} \not\in X_i$ for any $b > a_i$. This gives a
  $\kappa$-chain $(y_{a_i})_{i < \kappa}$ which must have an upper bound $y_{a_\kappa}$. By
  assumption, however, $y_{a_\kappa}$ must not be an element of any $X_i$, whence
  $y_{a_\kappa} \not\in Y$, a contradiction.
\end{proof}

\begin{theorem}[Raising the index of accessibility]
  \label{thm:accessible:jack}
  If $\EE$ is $\kappa$-accessible, and $\kappa \lhd \lambda$, then $\EE$ is $\lambda$-accessible.

  In fact, this can be strengthened to an if and only if~\parencite[Theorem
  2.11]{adamek-rosicky:1994}.
\end{theorem}
\begin{proof}
  First, it is immediate that $\EE$ has all $\lambda$-filtered colimits as these are a strict subset
  of $\kappa$-filtered colimits. It remains to show that $\EE$ is generated under $\lambda$-filtered
  colimits by some set of $\lambda$-compact elements.

  First, let us set $A$ to be the collection of all objects which are the $\kappa$-filtered colimits
  of $\kappa$-compact objects of size strictly smaller than $\lambda$. This collection objects is
  essentially small (since the collection of $\kappa$-compact objects is essentially small), and
  every object in $A$ is $\lambda$-compact.

  What remains is to show that every object of $\EE$ is a $\lambda$-filtered colimit of objects in
  $A$. We observe that any object $X$ can already be expressed as a $\kappa$-directed colimit
  $\Colim_{d \in D} X_d$ of $\kappa$-compact objects. Let us now take $\hat{D}$ to be the set of all
  $\kappa$-directed subsets of $D$ with cardinality smaller than $\lambda$. Using
  \cref{lem:accessible:sharp-directed-colim}, this subset is $\lambda$-directed. Moreover, it is we
  have that $\{{d} \in D\}$ is cofinal in $\hat{D}$, and that $X_S = \Colim_S X_d$ is a colimit of
  objects in $A$. Therefore, $\Colim_{S \in \hat{D}} X_S \cong \Colim_{d \in D} X_d \cong X$,
  exhibiting $X$ as a $\lambda$-directed colimit of objects in $A$ as required.
\end{proof}

\begin{corollary}
  \label{cor:accessible:index}
  If $\EE$ is an accessible category, it is $\kappa$-accessible for arbitrarily large $\kappa$.
\end{corollary}

\begin{corollary}
  \label{cor:accessible:uniform}
  If $\EE,\DD$ are accessible categories and $\Mor[F]{\EE}{\DD}$ is accessible, then there are
  arbitrarily large $\kappa$ such that $\EE,\DD$ and $F$ are all $\kappa$-accessible and $F$
  preserves $\kappa$-compact objects.
\end{corollary}
\begin{proof}
  The only interesting detail to prove is that an accessible functor preserves $\lambda$-compact
  objects for all sufficiently large $\lambda$, we can then take the maximum of all ordinals
  involved and use \cref{cor:accessible:index} to raise the indices of everything involved.

  Let us suppose that $\kappa$ is large enough to be the index of accesssiblity for $F$, $\EE$, and
  $\DD$. First, observe suppose that $K_i$ is $\kappa$-compact, then there exists some $\kappa_i$
  such that $F(K_i)$ is $\kappa_i$ presentable. Moreover, since there is a set of $\kappa$-compact
  objects, there is an ordinal $\kappa' \ge \kappa$ such that $F(K)$ is $\kappa'$-compact for all
  $\kappa$-compact $K : \EE$. Now, if $X$ is $\lambda$-compact in $\EE$, this means that it can be
  written as the retract of a $\kappa$-directed colimit of size $\lambda$ comprised of
  $\kappa$-compact objects. Therefore, $F(X)$ is a retract of a $\kappa$-directed colimit of size
  $\lambda$ comprised of $\kappa'$-compact objects. Therefore, $F(X)$ is $\lambda$-compact.
\end{proof}


\section{Gluing Grothendiek Toposes}
\label{sec:gluing}

In this section we prove the following theorem:
\begin{theorem}
  Given two Grothendieck toposes $\EE,\FF$ and an accessible lex functor $\Mor[F]{\EE}{\FF}$,
  $\GL{F}$ is a Grothendieck topos.
\end{theorem}
\begin{proof}
  It is proven in some detail of in \parencite[A2.1.12]{johnstone:2002} that $\GL{F}$ is an
  elementary topos if $F$ is lex. A more detailed proof of the construction of the subobject
  classifier is given by \textcite{gratzer:subobject-of-glued-topos:2020}. It remains to shwo that
  $\GL{F}$ is (1.) cocomplete (2.) accessible.

  For the first requirement, we will prove the stronger result that $\GL{F}$ is cocomplete whenever
  $\EE$ and $\FF$ are cocomplete. Suppose we are given a diagram $\Mor{D}{\GL{F}}$, there are two
  induced diagrams $\Mor{D}{\EE}$ and $\Mor{D}{\FF}$. Since these categories are cocomplete, we can
  form colimits for these diagrams $X = \Colim_i D_i$ and $\widebar{X} = \Colim_i
  \widebar{D}_i$. Moreover, by assumption there is a morphism $\Mor[d_i]{\widebar{D}_i}{F(D_i)}$,
  and these induce a morphism $\Mor[x]{\widebar{X}}{F(X)}$. We claim this is the colimit.  Suppose
  we are given $\Mor[y]{\widebar{Y}}{F(Y)}$ together with maps $\Mor[y_i]{d_i}{y}$:
  \[
    \DiagramSquare{
      nw = \widebar{D}_i,
      sw = F(D_i),
      ne = \widebar{Y},
      se = F(Y),
      ne/style = {right = 4cm of nw},
      east = y,
      west = d_i,
      south = F(y_i),
      north = \widebar{y}_i,
    }
  \]
  The condition on these maps ensure that they assemble into a pair of maps $\Mor[e]{X}{Y}$ and
  $\Mor[e]{\bar{X}}{\bar{Y}}$. We wish to show that these maps commute appropriately, and this
  follows from the universal property of $\bar{X}$.

  For the second requirement, we must show that $\GL{F}$ is accessible. First, let us use
  \cref{cor:accessible:uniform} to fix a $\kappa$ such that $\EE$, $\FF$ and $F$ are all $\kappa$
  accessible, and moreover that $F$ preserves $\kappa$-compact objects. Consider the following set
  of objects:
  \[
    S = \{\Mor[x]{\widebar{X}}{F(X)} \mid X,\Bar{X}\ \text{$\kappa$-compact}\}
  \]
  This is a set, because there is a set of $\kappa$-compact objects in $\EE$ and $\FF$ by
  accessibility. We first claim that this set contains $\kappa$-compact objects, which follows from
  the fact that $\widebar{X}$ and $F(X)$ are both $\kappa$-compact. Moreover, this set is closed
  under $\kappa$-colimits, because colimits are take pointwise. Therefore, if we can show that all
  objects are generated by $\kappa$-filtered colimits of objects from $S$, we will have completed
  the proof.

  Let us fix some arbitrary element $\Mor[x]{\widebar{X}}{X}$. By assumption, there are two
  $\kappa$-filtered diagrams, $(\widebar{D}_i)_{i \in I_0}$ and $(D_i)_{i \in I_1}$, such that
  $\widebar{X} = \Colim_i \widebar{D}_i$ and $X = \Colim_i D_i$. We write $\iota_i$ for the
  inclusion of $D_i$ into $X$ and $\widebar{\iota}_i$ for the inclusion of $\widebar{D}_i$ into
  $\widebar{X}$. Now, we take these two diagrams and define the following:
  \[
    (d_j)_{j \in J} = \{\Mor[d]{\widebar{D}_{i_0}}{D_{i_1}} \mid F(\iota_i) \circ d = x \circ \bar{\iota}_i \}
  \]
  There exists a morphism from $d_i$ to $d_j$ in this diagram precisely when the expected
  commutative square exists. Furthermore, using the fact that the underlying two diagrams are
  cofiltered, we observe that this diagram is cofiltered. Therefore, we are reduced to showing that
  $x$ is the colimit this diagram. There is a canonical cocone $\Mor{d_i}{x}$ built out of the
  cocones for $\widebar{X}$ and $X$, and because colimits are determined pointwise this cocone is
  universal.
\end{proof}


\printbibliography
\end{document}
