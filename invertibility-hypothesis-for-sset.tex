\documentclass{amsart}
\usepackage{preamble,categories,macros,}
\usepackage{locallabel,pftools}
\usepackage{jmsdelim}
\usepackage{jon-tikz}

\addbibresource{refs.bib}

\title{The Invertibility Hypothesis in \texorpdfstring{$\SSET$}{Simplicial Sets}}
\author{Daniel Gratzer}
\date{\today}


\begin{document}
\begin{abstract}
  A confusingly simple proof of the invertibility hypothesis for $\SSET$. The proof centers around
  an analysis of a few key homotopy pushouts.
\end{abstract}
\maketitle

Throughout this note, we will work with the left-proper combinatorial model structure described by
Lurie on categories enriched over some suitably nice monoidal model category~\parencite{lurie:2009}.

\begin{property}
  \label{prop:invertibility}
  Given a combinatorial left-proper monoidal model category $\SS$, we say that $\SS$ satisfies the
  invertibility hypothesis if, given an $\SS$-category $\CC$ and a cofibration
  $\EmbMor[f]{[1]_\SS}{\CC}$ whose image is an equivalence in $\CC$, the right-hand map of the
  following pushout is a weak equivalence:
  \[
    \DiagramSquare{
      nw = [1]_\SS,
      ne = \CC,
      sw = [1]_{\widetilde{\SS}},
      se = \CC[f^{-1}],
      se/style = pushout,
      north = f,
      north/style = embedding
    }
  \]
  Here, we make use of the distinguished $\SS$-categories $[1]_{\SS}$ and $[1]_{\widetilde{\SS}}$. Both
  have two objects $x$ and $y$. In $[1]_{\SS}$, there is precisely one morphism $\Mor{x}{y}$ and no
  morphism $\Mor{y}{x}$. In $[1]_{\widetilde{\SS}}$, pair of a homotopy inverse morphisms $\Mor{x}{y}$
  and $\Mor{y}{x}$.
\end{property}

Intuitively, this property says that when working with a simplicial category as a model for some
$(\infty,1)$-category, we may turn a weak equivalence into a true homotopy equivalence without
affecting the homotopy type of the category. Obviously, this could be very useful in practice since
it allows us a degree of strictness we would otherwise lack. For instance \textcite{lurie:2009}
crucially uses this property of obtain a direct characterization of the fibrations of model
structure on simplicial categories.

We are concerned with showing that $\SSET$ with the Kan-Quillen model structure enjoys this
property. The traditional reference for this is \parencite{dwyer:1980}, but this is slightly
unsatisfying for two reasons. Firstly, \textcite{dwyer:1980} are concerned only with simplicial
categories with a fixed set of objects, making it difficult to relate to the category of simplicial
categories. Additionally, the reference shows its age in its choice of terminology and technical
approaches.

The following proof seems to establish the desired result, but appears considerably simpler. To the
extend that I'm slightly suspicious it's correct!
\begin{theorem}
  $\SSET$ enjoys \cref{prop:invertibility}.
\end{theorem}
\begin{proof}
  Let us fix $\CC$ and $f$ as in the statement of \cref{prop:invertibility}.  Firstly, we follow the
  advice of \citeauthor{lurie:2009} restrict our attention to the case where $\CC$ is fibrant and
  split $\Mor{[1]_{\SSET}}{[1]_{\widetilde{\SSET}}}$ into a cofibration followed by a trivial
  fibration. In particular, we obtain some $E$ with $\EmbMor{[1]_{\SSET}}{E}$ such with
  $\CovMor[\sim]{E}{[1]_{\widetilde{\SSET}}}$. We can in fact choose $E$ to be the category with two
  objects $x$ and $y$, three morphisms $\Mor[f]{x}{y}$ and $\Mor[g,g']{y}{x}$ together with 2
  2-cells making $g$ left-inverse to $f$ and $g'$ right-inverse. This is not a fibrant object on its
  own, but this is unimportant.

  In fact, all that matters is that inclusion $\Mor{[1]_{\SSET}}{E}$ is a cofibration and the map
  $\Mor{E}{[1]_{\widetilde{\SSET}}}$ is a weak equivalence. The first map is a cofibration because
  it is injective on all hom-spaces and bijective on objects. Moreover, the map $\Mor{E}{[1]_\SSET}$
  is a weak equivalence because it's bijective on objects and a weak equivalence on each hom-set
  (inverses are unique up to homotopy, so each hom-space is contractible).

  Now, we first take the pushout of $f$ along $\EmbMor{[1]_{\SSET}}{E}$ to obtain the following:
  \[
    \DiagramSquare{
      nw = [1]_\SSET,
      ne = \CC,
      sw = E,
      se = \CC\gls{f^{-1}},
      east = e,
      se/style = pushout,
      north = f,
      north/style = embedding,
      south/style = embedding
    }
  \]
  In fact, $\Mor[e]{\CC}{\CC\gls{f^{-1}}}$ is a weak equivalence. We first observe that $e$ is
  bijective on objects. Next, it is an identity for every hom-space except
  $\Mor{\Hom[\CC]{x}{y}}{\Hom[\CC\gls{f^{-1}}]{x}{y}}$, which is attaches the extra left and right
  inverse to $f$. However, by assumption $f$ was already an equivalence, and thus possessed a left
  and right inverse already. Since inverses are unique up to homotopy, there is a deformation
  retract to this inclusion, so it is a weak equivalence of simplicial sets as well. This implies
  that $e$ is a weak equivalence.

  Next, we take the pushout of $\Mor{E}{\CC\gls{f^{-1}}}$ along $\Mor{E}{[1]_{\widetilde{\SSET}}}$:
  \[
    \DiagramSquare{
      nw = E,
      ne = \CC\gls{f^{-1}},
      sw = [1]_{\widetilde{\SSET}},
      se = \CC[f^{-1}],
      se/style = pushout,
      north = f,
      north/style = embedding,
    }
  \]
  Recall that simplicial sets are a left-proper model category, and therefore because
  $\Mor{E}{[1]_{\widetilde{\SSET}}}$ is a weak equivalence, so too is
  $\Mor{\CC\gls{f^{-1}}}{\CC[f^{-1}]}$. Therefore, by 2-out-of-three $\Mor{\CC}{\CC[f^{-1}]}$ is a
  weak equivalence.
\end{proof}

\printbibliography
\end{document}
