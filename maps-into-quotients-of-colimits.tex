\documentclass{amsart}
\usepackage{preamble}
\usepackage{jmsdelim}
\usepackage{jon-tikz}
\usepackage{macros}
\usepackage{categories}

\addbibresource{refs.bib}

\title{Subobjects of quotients of representables generate all monomorphisms}
\author{Daniel Gratzer}
\date{\today}

\begin{document}
\maketitle

Given a presheaf category $\PSH{\CC}$, We prove that the smallest saturated class of morphisms
containing all monomorphisms of the form $\EmbMor{X}{\Yo{C}/{\sim}}$ is all the class of all
monomorphisms. This fact is stated without proof in \textcite{cisinski:2019}.

First, we observe that if $\CC$ is small, because $\PSH{\CC}$ both well powered and cowell powered,
the collection of all subobjects of quotients of representables is essentially a set. Moreover,
monomorphisms are closed under retract, pushout, and transfinite composition in $\PSH{\CC}$. It
therefore suffices to show that any monomorphism $\EmbMor[m]{X}{Y}$ can be constructed out the set
of monomorphisms $\EmbMor{X}{\Yo{C}/{\sim}}$.

\begin{lemma}
  The collection of such monomorphisms is a set (written $L$).
\end{lemma}
\begin{proof}
  immediate because every topos is (co) well-powered and $\CC$ is a small category.
\end{proof}

\begin{lemma}
  Every element of $l(r(L))$ is a monomorphism.
\end{lemma}
\begin{proof}
  It suffices to show that monomorphisms are closed under (1) retracts (2) pushouts (3) transfinite
  compositions. The first is immediate, and the second is true in any topos. For the third, suppose
  we are given a transfinite sequence $\EmbMor[m_\alpha]{X_\alpha}{X_{1 + \alpha}}$,
  $\alpha < \lambda$ and that $f,g : \Mor{X}{Y}$ are equalized by $X_0 \to \Colim_\alpha
  X_\alpha$. Proceeding by induction on $\lambda$, we wish to show that $f = g$. Our goal is trivial
  if $\lambda = 0$. If $\lambda = 1 + \lambda'$, using the fact that $m_{\lambda'}$ is a
  monomorphism it suffices to show that $f \circ m_{\lambda'} = f \circ m_{\lambda'}$, assuming that
  they are equalized by $\Mor{X_0}{\Colim_{\alpha < \lambda'} X_\alpha}$ which follows directly from
  our induction hypothesis.

  Finally, we consider the case of limit ordinals. It suffices to show that for any
  $f \circ p = g \circ p$ for any morphism $\Mor[p]{\Yo{C}}{X}$, since representables form a
  generator for any presheaf category. Moreover $\Yo{C}$ since compact, we know that $p$ must factor
  through some $X_\beta$, $\beta + 1 < \lambda$. Our induction hypothesis tells us that
  $\Mor{X_0}{\Colim_{\alpha < \beta + 1} X_\alpha}$ is a monomorphism. Therefore, we may conclude
  that $f$ and $g$ are equalized by
  $\Mor{\Colim_{\alpha < \beta + 1} X_\alpha}{\Colim_{\alpha < \lambda}}$, and therefore that
  $f \circ p = g \circ p$ as required.
\end{proof}

\begin{lemma}
  If $m$ is a monomorphism, $m \in l(r(L))$.
\end{lemma}
\begin{proof}
  Suppose that we have $m : \EmbMor{X}{Y}$. We begin by fixing some ordinal $\lambda$ such that
  there and an epimorphism $e : \CovMor{\coprod_{\alpha < \lambda} \Yo{C_\alpha}}{Y}$ by taking by
  taking the discrete subcategory of the canonical diagram over $Y$. Let us now decompose $Y$ into a
  $\lambda$-chain $(Y_\alpha)_{\alpha < \lambda}$, where $Y_0 = X$ and $Y_{1 + \beta}$ contains all
  the points of $Y_\alpha$, for $\alpha \le \beta$, as well as the point $e_\beta$. This exhibits
  $m$ as a transfinite composition, so it suffices to show that $\Mor{Y_\alpha}{Y_{1 + \alpha}}$ is
  in $l(r(L))$.

  Factorize $e_{\alpha} : \Mor{\Yo{C_\alpha}}{Y}$ into an epi-mono and write
  $m_\alpha : \Mor{\Yo{C_\alpha}/{\sim}}{Y_\alpha}$. By definition,
  $Y_{1 + \alpha} = m_\alpha \cup Y_\alpha$ as a subobject of $Y$:
  \[
    \begin{tikzpicture}[diagram]
      \SpliceDiagramSquare{
        nw =  \Dom(m_\alpha) \cap Y_\alpha,
        ne = Y_\alpha,
        se = Y_{\alpha + 1},
        se/style = pushout,
        sw = \Dom(m_\alpha),
        north/style = embedding,
        south/style = embedding,
        east/style = embedding,
        west/style = embedding,
      };
      \node (Y) [below right = of se] {$Y$};
      \path[embedding, bend right] (sw) edge node[upright desc] {$m_\alpha$} (Y);
      \path[embedding, bend left] (ne) edge node {} (Y);
      \path[embedding, densely dotted] (se) edge node {} (Y);
    \end{tikzpicture}
  \]
  Accordingly, $\EmbMor{Y_\alpha}{Y_{1 + \alpha}}$ is the pushout of
  $\EmbMor{(\Yo{C_\alpha}/{\sim}) \cap Y_\alpha}{\Yo{C_\alpha}/{\sim}}$ along
  $\EmbMor{(\Yo{C_\alpha}/{\sim}) \cap Y_\alpha}{Y_\alpha}$. Accordingly,
  $\Mor{Y_\alpha}{Y_{1 + \alpha}}$ is the pushout of a morphism in $L$, and is therefore in
  $l(r(L))$.
\end{proof}

\begin{corollary}
  The class of monomorphisms in $\PSH{\CC}$ is generated by $l(r(L))$.
\end{corollary}

\printbibliography
\end{document}
